% sheet size
\geometry{a4paper, top=4.5cm, bottom=4.5cm, left=4cm, right=4cm} % heightrounded, bindingoffset=5mm}


% diamond itemize 
\setlist[itemize]{label=$\diamond$}


% list of figures
\graphicspath{ {figures/} }


% change from "Listing 1" to "Codice 1"
\addto\captionsitalian{%
\renewcommand{\lstlistingname}{Codice}}


% change from "Listings" to "Elenco dei codici"
\addto\captionsitalian{% 
\renewcommand{\lstlistlistingname}{Elenco dei codici}}


% create tree
\tikzset{
  treenode/.style = {shape=rectangle, rounded corners,
                     draw, align=center, 
                     top color=white, bottom color=blue!30},
  root/.style     = {treenode, font=\normalsize, bottom color=grey!50},
  env/.style      = {treenode, font=\small},
}



\titleformat{\chapter}[block]%
        {\normalfont\scshape\HUGE}%
        {\hspace*{-70pt}\color{BlueViolet!75}\fontencoding{U}\fontfamily{eur}\fontseries{b}\fontsize{60}{80}\selectfont\thechapter\hspace{15pt}}{10pt}
        {}[\chapterdecoration]

\newcommand\chapterdecoration{%
\begin{tikzpicture}[remember picture,overlay,shorten >= -10pt]

\coordinate (aux1) at ([yshift=-15pt]current page.north east);
\coordinate (aux2) at ([yshift=-410pt]current page.north east);
\coordinate (aux3) at ([xshift=-4.5cm]current page.north east);
\coordinate (aux4) at ([yshift=-150pt]current page.north east);

\begin{scope}[BlueViolet!50,line width=12pt,rounded corners=12pt]
\draw
  (aux1) -- coordinate (a)
  ++(225:5) --
  ++(-45:5.1) coordinate (b);
\draw[shorten <= -10pt]
  (aux3) --
  (a) --
  (aux1);
\draw[opacity=0.6,BlueViolet,shorten <= -10pt]
  (b) --
  ++(225:2.2) --
  ++(-45:2.2);
\end{scope}
\draw[BlueViolet,line width=8pt,rounded corners=8pt,shorten <= -10pt]
  (aux4) --
  ++(225:0.8) --
  ++(-45:0.8);
\begin{scope}[BlueViolet!70,line width=6pt,rounded corners=8pt]
\draw[shorten <= -10pt]
  (aux2) --
  ++(225:3) coordinate[pos=0.45] (c) --
  ++(-45:3.1);
\draw
  (aux2) --
  (c) --
  ++(135:2.5) --
  ++(45:2.5) --
  ++(-45:2.5) coordinate[pos=0.3] (d);   
\draw 
  (d) -- +(45:1);
\end{scope}
\end{tikzpicture}%
}


\everymath{\displaystyle}

\setlength{\parindent}{0em}
\renewcommand{\baselinestretch}{1.5}

% não permite separação silábica
\sloppy
\hyphenpenalty=100000
% não permite linhas orfãs e viúvas no tex
\clubpenalty=10000
\widowpenalty=10000
\displaywidowpenalty=10000

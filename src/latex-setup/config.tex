% Define your main color
\definecolor{MainColor}{RGB}{155, 0, 20} % Example RGB color for MainColor


% Diamond itemize list
\setlist[itemize]{label=$\diamond$}


% List of figures path
\graphicspath{{figures/}}


% Change from "Listing 1" to "Codice 1"
\addto\captionsitalian{%
    \renewcommand{\lstlistingname}{Codice}
}


% Change from "Listings" to "Elenco dei codici"
\addto\captionsitalian{%
    \renewcommand{\lstlistlistingname}{Elenco dei codici}
}


% Create tree style for TikZ
\tikzset{
    treenode/.style = {shape=rectangle, rounded corners,
                       draw, align=center, 
                       top color=white, bottom color=blue!30},
    root/.style     = {treenode, font=\normalsize, bottom color=grey!50},
    env/.style      = {treenode, font=\small},
}


% Customize the appearance of parts
\titleformat{\part}
    [display]         % display style for part title
    {\fontsize{20}{30}\bfseries}  % Format for part title
    {}                % Empty for part number
    {0pt}             % Space after part number
    {\begin{center} % Centering the content
        {\begin{tikzpicture}[remember picture, overlay]
            % Draw the hexagons
            \foreach \r in {7.0, 6.5, 6.0, 5.5, 5.0, 4.5} { % Hexagon radii
                % Calculate the position based on the radius
                \node[draw=MainColor!90, fill=white, regular polygon, regular polygon sides=6, minimum size=\r cm, inner sep=0pt, rotate=90, rounded corners] at (0 , 0) {}; % Adjusted position
            }
            % Place the part number inside the largest hexagon
            \node[scale=2, text=MainColor] at (0, 0) {\textbf{\partname\ \thepart}}; % Part title
            % Dynamically include the part title below the hexagons
            \node[scale=1.5, text=black] at (0, -3) {\fontsize{30}{30}\textbf{\titlename}}; % Custom title using the argument
        \end{tikzpicture}}  
    \end{center}}    % Custom hexagon for part title
[\vspace{5cm}]  % Space after the part title

% Create a new command to store the part title
\newcommand{\titlename}{}

% Modify \part command to save the title
\let\oldpart\part
\renewcommand{\part}[1]{%
  \renewcommand{\titlename}{#1} % Store the title
  \oldpart{} % Call the original part command
  \addcontentsline{toc}{part}{\titlename} 
}


% Customize the appearance of chapters
\titleformat{\chapter}
    [display]         % display style for chapter title
    {\Huge\bfseries}  % Format for chapter title
    {}                % Empty for chapter number (we will do it manually)
    {0pt}             % Space after chapter number
    {\begin{tikzpicture}[remember picture, overlay]
        % Draw the hexagon rotated by 90 degrees
        \node[draw=MainColor!90, fill=MainColor!90, regular polygon, regular polygon sides=6, minimum size=2.5cm, inner sep=0pt, text=white, rotate=90] at (-1.4, 0.4) {}; % Hexagon without number
        % Place the chapter number upright
        \node[text=white, scale=1.5] at (-1.4, 0.4) {\textbf{\thechapter}};
        % Horizontal line
        \draw[MainColor!90, thick] (-1.4, -0.22) -- (\textwidth, -0.22);
        % Draw little Hexagon at the endo of the line
        \node[draw=MainColor!90, fill=MainColor!90, regular polygon, regular polygon sides=6, minimum size=0.25cm, inner sep=0pt] at (\textwidth, -0.22) {};
    \end{tikzpicture}}    % Custom hexagon for chapter number
[\vspace{1cm}]        % Adjust space to bring the title closer to the hexagon




% Page style
\pagestyle{fancy}
\fancyhf{} % Clear header and footer
\fancyhead[L]{\leftmark} % Chapter/Section on left header
\fancyhead[R]{\thepage} % Page number on right header
